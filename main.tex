%%%%%%%%%%%%%%%%%%%%%%%%%%%%%%%%%%%%%%%
% Original author:
% Rahul Chauhan (http://rahulchauhan.net)
%
% Original repository:
% https://github.com/rahulworld/rahulworld-Resume
%%%%%%%%%%%%%%%%%%%%%%%%%%%%%%%%%%%%%%

\documentclass[]{karthik_sama}
\usepackage{fontawesome}
\usepackage[dvipsnames]{xcolor}
\usepackage[utf8]{inputenc}
\begin{document}

%%%%%%%%%%%%%%%%%%%%%%%%%%%%%%%%%%%%%%
%
%     COLUMN ONE
%
%%%%%%%%%%%%%%%%%%%%%%%%%%%%%%%%%%%%%%

\begin{minipage}[t]{0.33\textwidth} 
\begin{large}

\headername {Sama Sai Karthik\\}

\end{large}
\textit{\color{subs}Junior at IIIT-Bangalore\\}
\textit{IMtech program in Computer Science\\}

%%%%%%%%%%%%%%%%%%%%%%%%%%%%%%%%%%%%%%
%     About me
%%%%%%%%%%%%%%%%%%%%%%%%%%%%%%%%%%%%%%
\section{About Me} 
\noindent{\color{hr}\rule{5cm}{0.4pt}}\\
Always curious to get to the depths\\
of a subject, math incites this \\
thirst. Ardent animae lover. Willing\\
to unfold the baffling mystery - \\
{\lat \emph{How the mind works?}}

%%%%%%%%%%%%%%%%%%%%%%%%%%%%%%%%%%%%%%
%     SKILLS
%%%%%%%%%%%%%%%%%%%%%%%%%%%%%%%%%%%%%%
\section{Skills}
\noindent{\color{hr}\rule{5cm}{0.4pt}}
\subsection{Programming Languages}
C, C++, Python, Java, SQLite,Django\\
MiniZinc, Flutter, Bash, Javascript,\\
,Verilog, GDB, HTML, CSS, Bootstrap.
\vspace{1pt}
\subsection{OS}
Windows, GNU/Linux
\vspace{1pt}
\subsection{Other Skills}
Binary Exploitation, Reverse\\
Engineering, Beautiful penmanship
\sectionsep
%%%%%%%%%%%%%%%%%%%%%%%%%%%%%%%%%%%%%%
%     COURSEWORK
%%%%%%%%%%%%%%%%%%%%%%%%%%%%%%%%%%%%%%
\section{Coursework}
\noindent{\color{hr}\rule{5cm}{0.4pt}}
\subsection{Ongoing courses}
Automata theory and Computation\\
Machine Learning\\
Math for Machine Learning\\
Database Management Systems\\
Software Engineering 
\begingroup \renewcommand\baselinestretch{1.5}\subsection{Completed Courses}
\href{https://drive.google.com/file/d/1K62NvFlYC0u1oTgYgRi4-djtt6LelUNc/view?usp=sharing}{\faHandORight \space Transcript}
\endgroup

\begingroup \renewcommand\baselinestretch{1.5}\subsection{MOOCS}
Computational Neuroscience {\scriptsize(Ongoing)}\\
Machine Learning
\href{https://www.coursera.org/account/accomplishments/certificate/N5QW492YV7JP}{\faExternalLink}\\
Neural Networks and Deep Learning
\href{https://www.coursera.org/account/accomplishments/certificate/3VHZ3WCVM7PJ}{\faExternalLink}\\
Structuring Machine Learning Projects
\href{https://www.coursera.org/account/accomplishments/certificate/9UZNHSKKS6DZ}{\faExternalLink}\\
Improving Deep Neural Networks
\href{https://www.coursera.org/account/accomplishments/certificate/XKUNVU2PMN6P}{\faExternalLink}\\
\href{}{\faHandORight \space All other MOOCs completed}
\endgroup
\sectionsep

%%%%%%%%%%%%%%%%%%%%%%%%%%%%%%%%%%%%%%
%     EDUCATION
%%%%%%%%%%%%%%%%%%%%%%%%%%%%%%%%%%%%%%
\section{Education} 
\noindent{\color{hr}\rule{5cm}{0.4pt}}\\
\datecolor{2018-Present}
\subsection{IMtech in CSE}
IIIT Bangalore \\
CGPA : 3.84/4.00 \scriptsize{(4 semesters)}\\
\vspace{5pt}
\datecolor{2016-2018}
\subsection{Intermediate}
FIITJEE\\
Board exams percentage: 97.2\%\\
\vspace{5pt}
\datecolor{2010-2016}
\subsection{High School}
Nalanda Vidya Niketan, VJA\\
Class 10 GP : 10.0 \\
\sectionsep
%%%%%%%%%%%%%%%%%%%%%%%%%%%%%%%%%%%%%%
%
%     COLUMN TWO
%
%%%%%%%%%%%%%%%%%%%%%%%%%%%%%%%%%%%%%%

\end{minipage} 
\hfill
\begin{minipage}[t]{0.66\textwidth} 
% \descript{BS in Computer ence}
\hspace*{0pt}\hfill    \\
\hspace*{0pt}\hfill    \\
\hspace*{0pt}\hfill  \href{mailto:samasaikartik@gmail.com}{\faEnvelopeO \space samsaikartik@gmail.com}\\
\hspace*{0pt}\hfill \faMobile \space +91-7981915784 \\
\hspace*{0pt}\hfill \begin{tabular}{@{}c   c@{}}
     \href{https://www.linkedin.com/in/karthik-sama-18bb87193/}{\faLinkedin \space linkedin} & \href{https://github.com/Kartik-Sama}{\faGithub \space Github}\\
\end{tabular}
\hspace*{0pt}\hfill    \\
% \hspace*{0pt}\hfill Email.:\textbf{\href{mailto:rahulnitsxr@gmail.com}{rahulnitsxr@gmail.com}} \\
% \hspace*{0pt}\hfill Web.:\textbf{\href{http://rahulchauhan.net}{http://rahulchauhan.net}} 

%%%%%%%%%%%%%%%%%%%%%%%%%%%%%%%%%%%%%%
%     EXPERIENCE
%%%%%%%%%%%%%%%%%%%%%%%%%%%%%%%%%%%%%%
\section{Experience}
\noindent{\color{hr}\rule{12.5cm}{0.4pt}}
\datecolor{\footnotesize{Summer} 2020} \runsubsection{Multi Model Perception Lab, IIITB}
\descript{Research Intern}
\noindent
\hspace{5em}%
\begin{minipage}{0.85\textwidth\vspace{2pt}}
Indian sign language project: converting local Indian languages to sign language. I studied and 
experimented unsupervised and adversarial models for low resource machine translation, published by Facebook Research \href{https://github.com/facebookresearch}{\faExternalLink}.
\end{minipage}
\descriptright{PyTorch, MOSES, MUSE, FastText}
\sectionsep

\datecolor{\footnotesize{Winter} 2019 } \runsubsection{Kharagpur Winter of Code (KWoC)}
\descript{Contributor}
\noindent
\hspace{5em}%
\begin{minipage}{0.85\textwidth\vspace{2pt}}
Contributed the Kharagpur Winter of Code, received certificate of completion \href{https://drive.google.com/file/d/1cBdzGNHDSj6mLb5X-0RbTzSaoZFP7rzW/view?usp=sharing}{\faExternalLink}.I worked on a Django blog project. My contributions include adding features to ask questions on the platform, share blogs on other social media platforms and enabling assigning tags to various blogs.
\end{minipage}
\descriptright{Django, Github, SQLite}
\sectionsep

% \datecolor{20XX-now} \runsubsection{Indian Institute of Science, Banglore}
% \descript{Research Intern}
% \noindent
% \hspace{5em}%
% \begin{minipage}{0.85\textwidth\vspace{2pt}}
% pan sharpening is one kind of data fusion become very wide spread method which populate the spectral band information with the influence of high spatial information.
% \end{minipage}
% \descriptright{Satellite images(Chandrayan), Fusion Algorithm, Java}
% \sectionsep

% \datecolor{20XX-now} \runsubsection{Smokey, Banglore}
% \descript{Android Intern}
% \noindent
% \hspace{5em}%
% \begin{minipage}{0.85\textwidth\vspace{2pt}}
% Worked on location based services application which provide services at any location near
% services provider and also made friend module.
% \end{minipage}
% \descriptright{Android, Mysql, Google map}
%%%%%%%%%%%%%%%%%%%%%%%%%%%%%%%%%%%%%%
%     AWARDS
%%%%%%%%%%%%%%%%%%%%%%%%%%%%%%%%%%%%%%
\section{Achievements/Awards} 
\noindent{\color{hr}\rule{12.5cm}{0.4pt}}
\datecolor{2019-2020} \runsubsection{Dean's Merit List}
\descript{IIIT Bangalore}

% \hspace{5em}%
% \begin{minipage}{0.85\textwidth\vspace{2pt}}
% In this hackathon Developed distributed Voting Application.
% \end{minipage}
\noindent\datecolor{2018-2019} \runsubsection{Dean's Merit List}
\descript{IIIT Bangalore}
\datecolor{2018} \hspace{2.3em}\runsubsection{JEE Mains}
\descript{All India Rank 2344}
\noindent
\datecolor{2018} \hspace{2.3em}\runsubsection{JEE Advanced}
\descript{All India Rank 2654}
\noindent
\datecolor{2018} \hspace{2.3em}\runsubsection{ComedK}
\descript{All India Rank 11}
\noindent
\href{http://edu.saikarthik.in/}{\faHandORight \space All Olympiad achievements, Extra curricular activity achievements}
%%%%%%%%%%%%%%%%%%%%%%%%%%%%%%%%%%%%%%
%     SIDE PROJECT
%%%%%%%%%%%%%%%%%%%%%%%%%%%%%%%%%%%%%%
\vspace{1em}
\section{Project work}
\noindent{\color{hr}\rule{12.5cm}{0.4pt}}
\runsubsection{Off to Mars \footnotesize{(Django quiz interface)}}
\descript{Django, SQLite}
\noindent
A Django app built from scratch hosting questions from data base and validating them with answers to give results. The amount taken to give the quiz also timed, to arrange the leaderboard accordingly.\\ 
\vspace{0.6em}

\runsubsection{Analysing Yelp Dataset}
\descript{SQLite}
Analysed the Yelp Dataset using SQLite. In this project I analysed the data to infer what factors effect businesses getting better ratings/ reviews from customers. 
\vspace{0.6em}

\runsubsection{Lyrics Inn}
\descript{Python, Genius lyrics API}
Used python web scrapping from JSON objects retrieved by querying Genius lyrics API to get lyrics of desired songs on fly. Also the top 10 songs and their artists are displayed for user to pick if multiple query results found.
\vspace{0.6em}

\runsubsection{MIPS Simulator}
\descript{Verilog, GTKWave}
Modules for instruction Fetch, Decode, Execute, Memory, Write-back were made in separate Verilog files All these modules were integrated by a processor file to make the implementation extensible and easy to use.
\vspace{0.6em}

\runsubsection{Quiz Broadcast}
\descript{Python, Socket Programming}
A simple interface built on socket programming were a server can broadcast questions and by knowing the server IP one can connect and play the quiz.

\end{minipage} 
\end{document}  \documentclass[]{article}